% Options for packages loaded elsewhere
\PassOptionsToPackage{unicode}{hyperref}
\PassOptionsToPackage{hyphens}{url}
\PassOptionsToPackage{dvipsnames,svgnames,x11names}{xcolor}
%
\documentclass[
  letterpaper,
  DIV=11,
  numbers=noendperiod]{scrartcl}

\usepackage{amsmath,amssymb}
\usepackage{iftex}
\ifPDFTeX
  \usepackage[T1]{fontenc}
  \usepackage[utf8]{inputenc}
  \usepackage{textcomp} % provide euro and other symbols
\else % if luatex or xetex
  \usepackage{unicode-math}
  \defaultfontfeatures{Scale=MatchLowercase}
  \defaultfontfeatures[\rmfamily]{Ligatures=TeX,Scale=1}
\fi
\usepackage{lmodern}
\ifPDFTeX\else  
    % xetex/luatex font selection
\fi
% Use upquote if available, for straight quotes in verbatim environments
\IfFileExists{upquote.sty}{\usepackage{upquote}}{}
\IfFileExists{microtype.sty}{% use microtype if available
  \usepackage[]{microtype}
  \UseMicrotypeSet[protrusion]{basicmath} % disable protrusion for tt fonts
}{}
\makeatletter
\@ifundefined{KOMAClassName}{% if non-KOMA class
  \IfFileExists{parskip.sty}{%
    \usepackage{parskip}
  }{% else
    \setlength{\parindent}{0pt}
    \setlength{\parskip}{6pt plus 2pt minus 1pt}}
}{% if KOMA class
  \KOMAoptions{parskip=half}}
\makeatother
\usepackage{xcolor}
\setlength{\emergencystretch}{3em} % prevent overfull lines
\setcounter{secnumdepth}{5}
% Make \paragraph and \subparagraph free-standing
\ifx\paragraph\undefined\else
  \let\oldparagraph\paragraph
  \renewcommand{\paragraph}[1]{\oldparagraph{#1}\mbox{}}
\fi
\ifx\subparagraph\undefined\else
  \let\oldsubparagraph\subparagraph
  \renewcommand{\subparagraph}[1]{\oldsubparagraph{#1}\mbox{}}
\fi


\providecommand{\tightlist}{%
  \setlength{\itemsep}{0pt}\setlength{\parskip}{0pt}}\usepackage{longtable,booktabs,array}
\usepackage{calc} % for calculating minipage widths
% Correct order of tables after \paragraph or \subparagraph
\usepackage{etoolbox}
\makeatletter
\patchcmd\longtable{\par}{\if@noskipsec\mbox{}\fi\par}{}{}
\makeatother
% Allow footnotes in longtable head/foot
\IfFileExists{footnotehyper.sty}{\usepackage{footnotehyper}}{\usepackage{footnote}}
\makesavenoteenv{longtable}
\usepackage{graphicx}
\makeatletter
\def\maxwidth{\ifdim\Gin@nat@width>\linewidth\linewidth\else\Gin@nat@width\fi}
\def\maxheight{\ifdim\Gin@nat@height>\textheight\textheight\else\Gin@nat@height\fi}
\makeatother
% Scale images if necessary, so that they will not overflow the page
% margins by default, and it is still possible to overwrite the defaults
% using explicit options in \includegraphics[width, height, ...]{}
\setkeys{Gin}{width=\maxwidth,height=\maxheight,keepaspectratio}
% Set default figure placement to htbp
\makeatletter
\def\fps@figure{htbp}
\makeatother

\KOMAoption{captions}{tableheading}
\makeatletter
\makeatother
\makeatletter
\makeatother
\makeatletter
\@ifpackageloaded{caption}{}{\usepackage{caption}}
\AtBeginDocument{%
\ifdefined\contentsname
  \renewcommand*\contentsname{Table of contents}
\else
  \newcommand\contentsname{Table of contents}
\fi
\ifdefined\listfigurename
  \renewcommand*\listfigurename{List of Figures}
\else
  \newcommand\listfigurename{List of Figures}
\fi
\ifdefined\listtablename
  \renewcommand*\listtablename{List of Tables}
\else
  \newcommand\listtablename{List of Tables}
\fi
\ifdefined\figurename
  \renewcommand*\figurename{Figure}
\else
  \newcommand\figurename{Figure}
\fi
\ifdefined\tablename
  \renewcommand*\tablename{Table}
\else
  \newcommand\tablename{Table}
\fi
}
\@ifpackageloaded{float}{}{\usepackage{float}}
\floatstyle{ruled}
\@ifundefined{c@chapter}{\newfloat{codelisting}{h}{lop}}{\newfloat{codelisting}{h}{lop}[chapter]}
\floatname{codelisting}{Listing}
\newcommand*\listoflistings{\listof{codelisting}{List of Listings}}
\makeatother
\makeatletter
\@ifpackageloaded{caption}{}{\usepackage{caption}}
\@ifpackageloaded{subcaption}{}{\usepackage{subcaption}}
\makeatother
\makeatletter
\@ifpackageloaded{tcolorbox}{}{\usepackage[skins,breakable]{tcolorbox}}
\makeatother
\makeatletter
\@ifundefined{shadecolor}{\definecolor{shadecolor}{rgb}{.97, .97, .97}}
\makeatother
\makeatletter
\makeatother
\makeatletter
\makeatother
\ifLuaTeX
  \usepackage{selnolig}  % disable illegal ligatures
\fi
\IfFileExists{bookmark.sty}{\usepackage{bookmark}}{\usepackage{hyperref}}
\IfFileExists{xurl.sty}{\usepackage{xurl}}{} % add URL line breaks if available
\urlstyle{same} % disable monospaced font for URLs
\hypersetup{
  pdftitle={My title},
  pdfauthor={First author; Another author},
  colorlinks=true,
  linkcolor={blue},
  filecolor={Maroon},
  citecolor={Blue},
  urlcolor={Blue},
  pdfcreator={LaTeX via pandoc}}

\title{My title\thanks{Code and data are available at: LINK.}}
\usepackage{etoolbox}
\makeatletter
\providecommand{\subtitle}[1]{% add subtitle to \maketitle
  \apptocmd{\@title}{\par {\large #1 \par}}{}{}
}
\makeatother
\subtitle{My subtitle if needed}
\author{First author \and Another author}
\date{February 5, 2024}

\begin{document}
\maketitle
\begin{abstract}
First sentence. Second sentence. Third sentence. Fourth sentence.
\end{abstract}
\ifdefined\Shaded\renewenvironment{Shaded}{\begin{tcolorbox}[interior hidden, boxrule=0pt, borderline west={3pt}{0pt}{shadecolor}, breakable, enhanced, frame hidden, sharp corners]}{\end{tcolorbox}}\fi

Title: Exploring the Impact of Mode Effects on Nonresponse Rates in
Survey Research

Author: {[}Your Name{]}

Date: February 5, 2024

GitHub Repo: {[}Link to your GitHub repository{]}

Nonresponse rates are a significant concern in survey research as they
can introduce bias and affect the validity of study findings. In the
Special Virtual Issue on Nonresponse Rates and Nonresponse Adjustments
of the Journal of Survey Statistics and Methodology, various aspects of
nonresponse rates and adjustments are discussed to enhance the
understanding and management of this critical issue. One aspect that
warrants attention is the impact of mode effects on nonresponse rates.

Mode effects refer to differences in survey responses that arise from
the mode of data collection, such as face-to-face interviews, telephone
interviews, self-administered paper questionnaires, and online surveys.
These mode effects can influence both the likelihood of responding to a
survey and the content of responses, thus affecting nonresponse rates.
Understanding how mode effects contribute to nonresponse rates is
essential for designing effective survey methodologies and implementing
appropriate nonresponse adjustments.

Several studies have investigated the impact of mode effects on
nonresponse rates in survey research. For example, Couper et al.~(2007)
compared nonresponse rates across different survey modes and found
variations in response rates, with face-to-face interviews generally
yielding higher response rates compared to telephone and online surveys.
Similarly, Dillman et al.~(2009) conducted a meta-analysis of mode
effects and nonresponse rates, concluding that mode effects can
significantly influence response rates and data quality.

The presence of mode effects on nonresponse rates can be attributed to
various factors. One factor is the level of respondent engagement and
trust associated with different modes of data collection. Face-to-face
interviews may elicit higher response rates due to the interpersonal
interaction and perceived credibility of the interviewer. In contrast,
online surveys may face challenges in establishing trust and engagement,
leading to lower response rates (Bethlehem et al., 2011).

Additionally, the characteristics of the target population can interact
with mode effects to influence nonresponse rates. For instance, older
adults may prefer traditional modes of data collection, such as
face-to-face or telephone interviews, over online surveys, leading to
differential nonresponse rates across age groups (Vannieuwenhuyze et
al., 2016). Similarly, individuals with limited access to technology or
low digital literacy may be less likely to respond to online surveys,
contributing to mode-related nonresponse bias (Callegaro \& DiSogra,
2008).

Addressing mode effects on nonresponse rates requires careful
consideration of survey design, mode selection, and nonresponse
adjustment techniques. Researchers may employ mixed-mode survey designs
to capitalize on the strengths of different modes while mitigating
mode-related biases (Brick \& Williams, 2013). Additionally, strategies
such as adaptive survey designs, personalized invitations, and
incentives can help improve response rates across diverse modes of data
collection (Singer et al., 2014).

In conclusion, understanding the impact of mode effects on nonresponse
rates is essential for ensuring the quality and representativeness of
survey data. By examining the literature on mode effects and nonresponse
rates, researchers can identify factors contributing to mode-related
biases and implement effective strategies to minimize nonresponse bias
in survey research.

References:

Bethlehem, J., Biffignandi, S., \& Pedersen, R. (2011). Improving survey
response: Lessons learned from the European Social Survey. John Wiley \&
Sons. Brick, J. M., \& Williams, D. (2013). Explaining rising
nonresponse rates in cross-sectional surveys. The ANNALS of the American
Academy of Political and Social Science, 645(1), 36-59. Callegaro, M.,
\& DiSogra, C. (2008). Computing response metrics for online panels.
Public Opinion Quarterly, 72(5), 1008-1032. Couper, M. P., Kapteyn, A.,
Schonlau, M., \& Winter, J. (2007). Noncoverage and nonresponse in an
Internet survey. Social Science Research, 36(1), 131-148. Dillman, D.
A., Smyth, J. D., \& Christian, L. M. (2009). Internet, mail, and
mixed-mode surveys: The tailored design method. John Wiley \& Sons.
Singer, E., Van Hoewyk, J., \& Maher, M. P. (2014). Experiments with
incentives in telephone surveys. Public Opinion Quarterly, 78(3),
747-770. Vannieuwenhuyze, J. T., Loosveldt, G., \& Molenberghs, G.
(2016). The impact of mode on response rates and bias in survey
research. Survey Practice, 9(3). \# Introduction

You can and should cross-reference sections and sub-sections.

The remainder of this paper is structured as follows.
Section~\ref{sec-data}\ldots.

\hypertarget{sec-data}{%
\section{Data}\label{sec-data}}

\hypertarget{results}{%
\section{Results}\label{results}}

\hypertarget{discussion}{%
\section{Discussion}\label{discussion}}

\newpage

\hypertarget{references}{%
\section{References}\label{references}}



\end{document}
